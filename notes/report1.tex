\enableregime [utf-8]
\mainlanguage [en]
\setupbodyfontenvironment[default][em=italic]
\setupwhitespace[medium]

\starttext
\title{Design of library for controlling a liquid handling robot}

\section{Informal notes regarding RoboEase}
\startitemize
\item My current focus is on creating a good programming library rather than an entire scripting language.  This allows us to leverage the work already put into existing frameworks.
\item The library is currently written in Scala (a modern, mature, java-compatible language).
\item Existing RoboEase scripts will be parsed and handled by the library.
\item The library approach allows for a greater degree of abstraction and modularity than we could achieve by further developing our own scripting language (due to the amount of time that would need to be invested in the new language to achieve such qualities).
\item In terms of RoboEase's level of abstraction, it fits between what will be described below as the level 1 and level 2 ASTs.
\stopitemize

\section{Components}
These following components act together to generate a concrete sequence of robot instructions:
\startitemize
\item {\em Generic Software Module}: a library of routines for processing all aspects of our robotics language except for those functions which are robot-specific.
\item {\em Robot Plugin Module}: a plugin for the Generic Software Module that has been tuned to the particularities of a specific robot.
\item {\em Robot Configuration}: information about robot's parts and capabilities.
\item {\em Peripheral Plugin Modules}: interfaces for controlling robot peripherals.
\item {\em Generic Protocol}: a generic protocol to be performed.  It should be described in such general terms that it can be translated into a concrete series of instructions for any number of liquid handling robots.
\item {\em User Parameters}: parameters the user must set in order to generate a concrete sequence of instructions from the generic protocol.
\item {\em User Overrides}: optional overrides of default choices made by the program.
\stopitemize

\section{Levels of parsing}
Protocol commands are parsed to create an Abstract Syntax Tree (AST), which is subsequently transformed into a series of increasingly concrete ASTs, until the final AST can be translated directly into a sequence of instructions for the target robot.  We will progress through 5 phases of AST concretization, from Level 4 down to Level 0, which differentiate themselves as follows:

\startitemize
\item Level 4 AST: represents the abstract protocol and the robot configuration, but it may be incomplete due to missing parameters.
\item Level 3 AST: incorporates the user parameters, and is therefore sufficient as a basis for ultimately creating concrete output.
\item Level 2 AST: incorporates optional overrides of default values.  All variables have now been fixed.
\item Level 1 AST: composed of only a small, basic instruction set, appropriate for translation into the language of the target robot.
\item Level 0 AST: instructions in the language of the target robot.
\stopitemize

The steps of translation are described here:

\startitemize[n]
\item
 {\em Input}: generic protocol, robot configuration

 {\em Output}: Level 4 AST, list of required parameters for the user to set

\item
 {\em Input}: Level 4 AST, user parameters

 {\em Output}: Level 3 AST, list of optional user overrides

\item
 {\em Input}: Level 3 AST, user overrides

 {\em Output}: Level 2 AST, an updated list of optional user overrides

\item
 {\em Input}: Level 2 AST

 {\em Processing}: expansion of high-level commands into our basic command set, with optimization.

 {\em Output}: Level 1 AST

\item
 {\em Input}: Level 1 AST

 {\em Processing}: the basic commands are translated into a program for the target robot.

 {\em Output}: Level 0 AST

\item
 {\em Input}: Level 0 AST

 {\em Output}: output for the target robot (e.g. an .esc script for Tecan robots)
\stopitemize


\section{Levels of command abstraction}

We distinguish between three types of commands, as described below.
2)
2) an extensible higher-level set of commands which can make use of both basic commands as well as other commands from the higher-level set.  
By the time our parsing precedure has reached the Level 1 AST, all higher-level commands have been translated into a sequence of basic commands.

\subsubsection{Basic command set}
The {\em basic command set} is a small set of core commands which every robot is expected to support.  
The Robot Plugin will need to translate these commands into its own format.  Currently these command are:
\startitemize
\item {\em Aspirate}: aspirate a liquid into a tip
\item {\em Dispense}: dispense a liquid from a tip into a destination well
\item {\em Clean}: clean the tips; depending on the robot, this might mean disposal or washing
\item {\em Mix}: use a tip to mix the liquid in a well by aspirating and re-dispensing 
\item {\em Wait}: wait for a given duration before proceeding with further commands
\item {\em Transfer}: move a plate from one location to another
\stopitemize

\subsubsection{Peripheral command set}
The {\em peripheral command set} is an extensible set of commands which are made available by Peripheral Plugin Modules.
The peripheral plugin modules provide both low-level and high-level commands for their peripheral of interest.  Here are several examples of peripherals which I am currently working on:
\startitemize
\item {\em Shake}: place a plate on a shaker
\item {\em Seal}: seal a plate
\item {\em Peel}: peel a sealed plate
\item {\em PCR}: perform PCR
\item {\em Centrifuge}: centrifuge a plate or tube
\stopitemize

\subsubsection{Higher-level commands}
These commands ease the task of programming by removing the need for the user to make detailed choices.  For example, the pipetting commands automatically choose which tips to use and can decide how to avoid contamination.
Here are a few examples:

\startitemize
\item {\em Pipette}: transfer a certain volume from a source well to a destination well; avoid contamination by performing cleaning operations at the appropriate times.  This command uses the low-level {\em Clean}, {\em Aspirate}, and {\em Dispense} commands.
\item {\em Copy plate}: transfer a certain amount of liquid from each well on one plate to the corresponding well on another plate.  This uses the {\em Pipette} command.
\item {\em Plate}: put several drops of a liquid onto a larger surface.
\stopitemize

[TODO: expand list of higher-level commands and describe which abstractions they provide]

\stoptext

